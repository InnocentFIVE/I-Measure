\section{测\kern\ccwd 度}
\begin{quote}
    先有\,$\sigma $-代数还是先有测度? 如果先有\,$\sigma $-代数的话, 测度空间就要叫做\,$\sigma $-代数空间了!
\end{quote}
考虑到测度可以建立在其他各种各样的结构上, 我们自然认为先有测度再有\;$\sigma $-代数, 这(看起来)也符合历史的发展轨迹. 但是在引入测度的过程中我们不得不反而道而行之.

\begin{defi}[($\sigma $-代数上的)测度]
    令 $X$ 是集合, $\mathcal M$ 是 $X$ 上的$\sigma $-代数, 一个定义在 $\mathcal M$ 上的测度是这样一个函数 $\mu :\mathcal M\to [\,0,\infty\,]$ 满足
    \begin{itemize}
        \item $\mu (\varnothing)=0$;
        \item $\{A_n\}_{n\geqslant 0}$ 两两不交, $\mu (\bigcup_{n\geqslant 0} A_n) = \sum_{n\geqslant 0} \mu (A_n)$.
    \end{itemize}
\end{defi}
暂且和\hyperref[测度的性质]{测度的性质 Ver.2} 一致. 一个首先要担心的是: 这样的函数存不存在. 显然让 $\mu\equiv 0$ 是一个平凡的测度, 这里引入一个不平凡的但是直接在幂集中处理的测度: Dirac 测度.
\begin{defi}[Dirac 测度]
    令 $X$ 是集合, 固定 $x_0\in X$, $\delta:\mathcal P(X)\to \left\{ 0,1 \right\} $ 满足
    \[
        \delta (E)=\begin{cases}
            1, & x_0\in E,    \\
            0, & x_0\notin E.
        \end{cases}
    \]
    这是一个 Dirac 测度.
\end{defi}
Dirac 测度看上去简单, 但还是有用的. 对 Dirac 测度的积分可以视为是 Dirac $\delta $ 函数的严谨描述.

存在性暂时解决了之后,剩下的就是关注测度的``值分布'', 我们关心的主要内容是这个函数会不会摸到 $\infty$. 显然 $\mu (X)$ 是最大的那个家伙, 如果 $\mu (X)$ 有限, 那么我们称这个测度为``\textbf{有限的}''; 如果 $\mu (X)=\infty$, 但是其可以分划成可数个\enote 测度有限集合之并, 则称测度为``\textbf{$\boldsymbol\sigma $-有限的}''. 事实上你可以这样认为, $\sigma $-有限的测度其实就是可数个有限的测度加起来. 虽然很容易攒到无穷, 但是有些性质是可以通过这种``加起来''从有限测度传递到\;$\sigma $-有限测度的\enote\enote.


现在我们要讨论的是测度的连续性------即使我们没有在 $X$ 上面赋予任何拓扑, 我们现在考虑的应当是``集合的极限''.

\begin{defi}[集合的极限]
    我们知道, $\mathbb R$ 中有这么一个序关系, 以及序列的上下极限:
    \[
        \liminf_{j\geqslant 0} a_j = \adjustlimits\sup_{n\geqslant 0}\inf_{j\geqslant n}a_j,\quad\limsup_{j\geqslant 0}a_j=\adjustlimits\inf_{n\geqslant 0}\sup_{j\geqslant n}a_j
        .\]
    考虑到 $\mathbb R$ 中是以 $\leqslant $ 为序关系的. 那么不如在 $\mathcal P(X)$ 是引入以 $\subseteq$ 的序关系. 这样一来对集列 $\{A_n\}_{n\geqslant 0}$. 其 $\sup$ 自然要大于所有集合, 故包含所有集合中的元素, 即:
    \[
        \sup_{n\geqslant 0}A_n =\bigcup_{n\geqslant 0} A_n,\,\text{\kaishu 同理}\,    \inf_{n\geqslant 0}A_n =\bigcap_{n\geqslant 0} A_n\mbox{\enote}.
    \]
    直接把符号代如上下极限得到:
    \[
        \liminf_{j\geqslant 0} A_j = \adjustlimits\bigcup_{n\geqslant 0}\bigcap_{j\geqslant n}A_j,\quad\limsup_{j\geqslant 0} A_j=\adjustlimits\bigcap_{n\geqslant 0}\bigcup_{j\geqslant n}A_j.
    \]
    如果一个集列真有极限, 那么只能是 $\liminf_{j\geqslant 0} A_j=\limsup_{j\geqslant 0} A_j$. 一般上升集或者下降集(就是逐步变大 / 变小的集合)极限是自然存在的, 分别为 $\bigcup_{j\geqslant 0} A_j$ 与 $\bigcap_{j\geqslant 0} A_j$.
\end{defi}
在此情形, 我们就可以讨论测度的连续性. 我们先从比较简单的上升 / 下降集开始.

\paragraph{测度的连续性}\mbox{}\kern-\ccwd\enote
\begin{itemize}
    \item 若 $\{E_n\}_{n\geqslant 0}\uparrow_n$(后文用此来表示``上升''), 则
          \[
              \mu \biggl(\, \bigcup_{n\geqslant 0}E_n \biggr) = \mu  \biggl(\, \bigcup_{n\geqslant 0}(E_n\setminus E_{n-1})\, \biggr) = \sum_{n\geqslant 0} \mu (E_n\setminus E_{n-1}) = \lim_{n \to \infty} \mu (E_n).
          \]
    \item 若集列下降, 状况有变: $\bigcup_{n\geqslant 0}(n,\infty) = \varnothing$, 但若将 $\mu $ 视为 $\mathbb R$ 中的长度测度, 则 $\lim_{n \to \infty} \mu((n,\infty))=\infty $. 矛盾! 但理论上, 集列的下降亦是其补集之上升, 即 \label{测度的连续性:集列下降}
          \[
              \begin{aligned}
                  \mu (X)-\lim_{n \to \infty} \mu (E_n) & = \lim_{n \to \infty} \mu (E_n^\complement)= \mu \biggl(\,\bigcup_{n\geqslant 0}E_n^\complement\,\biggr) \\
                                                        & = \mu(X) -  \mu \biggl(\,\bigcap_{n\geqslant 0}E_n\biggr)                                                \\
                                                        & \!\implies \lim_{n \to \infty}  \mu (E_n)=\mu \biggl(\,\bigcap_{n\geqslant 0}E_n\biggr)
              \end{aligned}
          \]
          但此式仅在 $\mu (X)<\infty$ 才能保证成立. 同理, 用 $\bigcup_{n\geqslant 0}E_n=E_1$ 代替 $X$ 也是一样的结果.
    \item 现在我们讨论上下极限, 这意味着我们要考虑更一般的情形. 令 $\{E_n\}_{n\geqslant 0}$ 是集列, 则 $\liminf_{j\geqslant 0} \mu (E_j)\geqslant \mu (\liminf_{j\geqslant 0} E_j)$, 我们管这叫下半连续性. 反之, 考虑 $\mu (\bigcup_{n\geqslant 0}E_j)$ 有限情形亦有 $\mu (\limsup_{j\geqslant 0} E_j)  \geqslant \limsup _{j\geqslant 0}\mu (E_j)$, 即下半连续性.

          证明思路其实很简单, 只需把 $\liminf$, $\limsup$ 之流用 $\sup$, $\inf$ 等价替换即可:
          \begin{gather*}
              \mu \Bigl(\liminf_{j\geqslant 0} E_j\Bigr)=\mu \biggl(\,\adjustlimits\bigcup_{n\geqslant 0}\bigcap_{j\geqslant n}E_n\biggr) = \lim_{n \to \infty}\mu \biggl(\,\bigcap_{j\geqslant n}E_n\biggr), \\
              \liminf_{j\geqslant 0}\mu(E_j) = \adjustlimits\sup_{n\geqslant 0}\inf_{j\geqslant n} \mu(E_j).
          \end{gather*}
          由 $\bigcap_{j\geqslant n}E_n\subseteq E_j$, $(\forall j\geqslant n)$. 故 $\mu (\bigcap_{j\geqslant n}E_n)\leqslant \inf_{j\geqslant n}\mu(E_j)$. 取上确界后亦然, 仿照第二款可证明 $\limsup$ 情形.
\end{itemize}
考虑连续性这种看起来有定量风范的性质后, 我们不妨考虑两个集合之间的误差, 用``对称差''以记之:
\begin{defi}[对称差]
    $A$, $B$ 是集合, 则 $A$, $B$ 的对称差 $A\dif B$ 定义为 $(A\setminus B)\cup (B\setminus A)$.
\end{defi}
在 $A$, $B$ 大部分重合的情形套以测度以描述误差: 令 $A$, $B$ 测度有限, 则 $\mu (A\dif B)=0\implies \mu (A)=\mu (B)$. 特别地, 描述两个集合之对称差亦是一种技巧: 不难看出是在欲证两个集合逼近程度最坏情形的描述.