\tableofcontents
\begin{tikzpicture}[remember picture,overlay]
    \draw[line cap=round, ->, white!10!black, thick] (zero) to (i);
    \draw[line cap=round, ->, white!20!black, thick] (i) to (ii);
    \draw[line cap=round, ->, white!30!black, thick] (ii) to (iii);
    \draw[line cap=round, ->, white!40!black, thick] (iii) to (iv);
    \draw[line cap=round, ->, white!50!black, thick] (iv) to (v);
    \draw[line cap=round, ->, white!60!black, thick] (v) to (vi);
    \draw[line cap=round, ->, white!70!black, thick] (vi) to (vii);
\end{tikzpicture}
\definecolor{DarkTurquoise}{rgb}{0, .48, .48}
\clearpage
\begin{abstract}
    这是一篇科普风格的文章. 因此对于文中所述定理, 我并不会全部给出证明\enote . 大多数具体的证明可见 \cite{Folland99}.

    这篇文章的本意是想要把一些测度的构造说清楚, 诚然, 测度的构造有许多种路径, 本文只是选取笔者最熟悉的一种路径, 同时对构造中所遇障碍以及对测度的积分进行一些阐述: 第一到第六节循序渐进地阐述了标准的实分析内容: 测度的构造, 延拓和 Lebesgue 测度; 第七节按照惯例阐述对测度的积分, 但自然而然地, 对积分的介绍不可避免地变得相当冗长, 几乎用了全文的一半内容才堪堪完成任务. 本文另一个特殊的点是注释繁多: 主要用于介绍测度相关的其余内容和一些对积分, 求导的处理, 终究是实分析的主干内容之一, 但因各种原因含量远超预算, 此主要由于笔者写作风格随性所致. 虽言``写科普要做减法'', 但笔者无法合理地处理注释的删减, 因此在此版本中保留了所有注释, 并将其置于文末以便提升正文阅读体验. 虽然, 但笔者仍然认为注释是本文中相当有趣的一部分, 笔者在此记录了不少自身的理解以及某些板块之间的联系(以及\hyperref[可爱猫猫]{可爱猫猫的图片}), 某种意义上也算是(超大量的)饭后甜品, 食之无妨.

    因笔者能力有限, 故差错难以避免, 还望读者海涵.
\end{abstract}