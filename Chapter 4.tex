\section{不测之忧}
\begin{quote}
    这些集合到底是什么? 它们既不是可测的集合也不是零测集. 难道我们不能把它们称为零测集骨架的鬼魂吗?
\end{quote}

事实上, 不可测集并非如此深秘, 结果上来说, 它只是不包含在我们的``可测''集合中, 或者处理这个集合容易与测度的定义发生矛盾. Banach\,--\,Tarski 定理就是这样一个著名的例子. 我们这里举另外一例不测之忧\enote:

我们先从直观上假想一下 $\mathbb R$ 上的 Lebesgue 测度 $m$ 大概就是我们所称的``长度'', 且满足传递不变性. 则 $m([\,0,1))=1$. 一个经典构造不可测集的方法是将一个有限正测度集合 $A$ 分解成可数个集合 $\{E_n\}_{n\geqslant 0}$ 的不交并. 其中通过某些处理(比如传递不变性)使得各 $E_n$ 测度(如果有)一致. 但如果真的这样做的话, $m(A)=\sum_{n\geqslant 0}m(E_n)$. $m(E_n)=0\implies m(A)=0$, $m(E_n)>0\implies m(A)=\infty$, 无论怎样都是不合理的. 因此可测集只能将此类 $E_n$ 排除在外.

为了方便处理, 我们将 $[\,0,1)$ 首尾相接形成一个圆 $C$, 且同时 $m(C)=1$\enote. 考虑圆上``有理旋转'': $\Pi _r:C\to C$, 将整个圆上的点旋转 $r\pi$ 的变换, 其中 $r\in\mathbb Q$. 若圆上两个点是等价的, 当且仅当其可以互相通过有理变换迁移, 也就是角度相差了 $\pi $ 的有理数倍.

不难发现, 圆上有很多这样等价的点, 把所有和 $a$ 点等价的点用集合包起来记为 $[\,a\,]$, 意为 $a$ 的等价类. 同时, 圆上有很多这样的等价类(事实上有不可数个), 直观理解大概是实数商掉有理数得到的``无理数元'':
\[
    C = [\,0\,]\cup[\,\sqrt{2}\,]\cup[\,\mathrm e\,]\cup[\,1/\pi \,]\mkern-10mu\underbrace{~\cup~\cdots~}_{\textit{其实这里有不可数个}}
\]
另一个直观看法是, 两个不相同的等价类之间相差了一个``无理旋转''. 现在从这么些不可数个等价类里每个取一个元素\enote 并成一个集合 $N$. 显然这个集合不可数, 事实上, 这个 $N$ 就是所有的``无理数元''.

考虑到实数就是有理数补充上无理数, 我们发现``有理旋转''会改变 $N$, 也即: $\Pi _r(N)\neq N$. 因为 $N$ 是``无理数元'', 因此若加上一个有理旋转并不能抵消, 如果加上一个``无理旋转'', 由于旋转量在 $N$ 中有对应的元反而可以抵消. 对任何有理数都如此.

故 $\Pi _r(N)$, $(r\in\mathbb Q)$ 两两不交. 且
\[
    C = \bigcup_{r\in\mathbb Q}\Pi _r(N)
    .\]
这个看似并不意外的结论来自于任意实数去除自身的``无理数元''部分后必然是个有理数. 因此可以用 $N$ 复合这个有理数对应的旋转来摸到这个实数.

接下来我们发现, $m(N) = m(\Pi _r(N))$, (由于只是普通的旋转, 或者实轴中的普通平移, 自然不改变 Lebesgue 测度). 这意味着我们把 $C$ 分解为了可数个两两不交相同测度集合之并. 依赖于我们一开始的阐述, $N$ 只能是不可测集, $\Pi _x(N)$, $(\forall x\in\mathbb R)$ 连坐.

这个例子可能还好, 毕竟一看就知道是为了捣乱才凹出来的.

但事实上, 对以一个零测度的集合, 我们会下意识地认为其子集必然可测, 且测度为 $0$, 而实际上不然\enote. 虽然其是容易解决的, 但其过程某种意义上有些繁琐: 一个造成其繁琐的原因是我们总有办法让这个零测集子集某种意义上``可测'', 但会引起某些信息的改变, 见后文测度完备化的内容. 不过这终究启迪我们, 零测集之中可能大有乾坤, 对待这些事物必然不能总是按照直觉考虑.